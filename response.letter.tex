\documentclass[journal,onecolumn,12pt]{IEEEtran} 

\usepackage{amsmath,amssymb,bm}
\usepackage{amsthm, amsfonts}	
\usepackage{bm,bbm}
\usepackage[normalem]{ulem}
\usepackage{color}
\usepackage{fancybox}
\usepackage{url,booktabs}

\usepackage{xr}



\title{Reply to Reviewer's Comments on\\
``EEG characterization and classification based on image gradient histograms''\\
(now ``Oriented Gradient Histogram applied to P300 Detection'')}
\author{}

\begin{document}

\maketitle
\pagenumbering{roman}
\setcounter{page}{1}

We are grateful to the reviewers for pointing at relevant issues in our manuscript.

In the following, we discuss how we dealt with each raised issue. 

\vskip+1ex
\noindent \dotfill

\section*{\fbox{Reviewer \#1:}}

I have read the manuscript "EEG characterization and classification based on Histogram of Gradients of Signal Plots" with great interest since its title promised a solution to an important problem. In spite of presenting an interesting methodology and some interesting issues related to EEG characterization, I consider the manuscript is not ready for publication. 

First of all, the title is misleading since a first, and naive, read suggest being able to classify and characterize any EEG with their different components (i.e., P200, P300, P600, N100, N200, N400 to name a few). Clearly, the title does not match the content since the authors focus on the P300; of course, there is evidence of the benefits of working with these signals, but, is the proposed method extensible to other signals?, this is not clear from the manuscript. 

Another issue with the manuscript is its mathematical vagueness. For example, the variable T is used as a vector or two-tuple but its notation is different from the notation the authors use for vectors. Moreover, there is something wrong in equation (4) and something inconsistent between equations (7) and (8). Also, the authors go back and forth the digital space and the continuous space and although there is an equivalence, the manuscript could benefit from a clearer explanation. 

Another issue that I find worrisome is the design of experiments. First of all, it seems that the number of subjects is rather low, it would be interesting to know why the authors decided to use such a subset. Also, the authors speculate that their results are not that good because they did not take into account the latency and magnitude of the signals, there is previous evidence in the literature that the P300 suffers from these, and other, problems and, thus, one wonders why the authors did not anticipate such problems. 

A couple of finale observations. First, it seems the authors assume that the readers have some previous deep knowledge of some topics such as that some channels should carry more or better information without a previous reference. Another issue has to do with the way the manuscript is written, sometimes is very difficult to follow the ideas the authors want to convey, an example can be found in the paragraph containing equation (5), the manuscript surely benefit from a careful review before submitting.


\subsection*{\ovalbox{General Comments}}
I have read the manuscript "EEG characterization and classification based on Histogram of Gradients of Signal Plots" with great interest since its title promised a solution to an important problem. In spite of presenting an interesting methodology and some interesting issues related to EEG characterization, I consider the manuscript is not ready for publication. 
\vskip+1ex
\noindent \dotfill
\vskip+1ex
%
\begin{quotation}
{\color{blue}
Thank you for your feedback. We have modified several points according to your recommendations, we hope the article is now better suited for publication. 
}
\end{quotation}
%
\vskip+1ex
\noindent \dotfill
\vskip+1ex

2.1) the title is misleading since a first, and naive, read suggest being able to classify and characterize any EEG with their different components (i.e., P200, P300, P600, N100, N200, N400 to name a few). Clearly, the title does not match the content since the authors focus on the P300

\vskip+1ex
\noindent \dotfill
\vskip+1ex
%
\begin{quotation}
{\color{blue}
Thank you for the observation. The title of the article was changed accordingly to your suggestion.  
}
\end{quotation}
%
\vskip+1ex
\noindent \dotfill
\vskip+1ex

2.2) there is evidence of the benefits of working with these signals P300, but, is the proposed method extensible to other signals?, this is not clear from the manuscript. 


\vskip+1ex
\noindent \dotfill
\vskip+1ex
%
\begin{quotation}
{\color{blue}
You're  right, thank you very much for your comment.  We published in Ramele et al 2016 the application of the same method to identify rhythmic EEG events like Visual Occipital Alpha Waves and to decode Motor Imagery.  We are also working on unpublished material where we are analyzing the same approach for the detection of K-Complexes and to decode SSVEP patterns. Modificar%We modified the article to emphasize more clearly that the presented algorithm is  applicable to P300.
 
% Como haces para que las referencias que estan aca en la respuesta apunten al articulo ???
%An explanation  of which is a satisfactory result for Algorithm~\ref{Algoritmo} was added in sections~\ref{errorevaluation} and~\ref{sec:EdgeSimulatedData}.  
}
\end{quotation}
%
\vskip+1ex
\noindent \dotfill
\vskip+1ex
\subsection*{\ovalbox{Third Paragraph}}

3.1) Another issue with the manuscript is its mathematical vagueness. For example, the variable T is used as a vector or two-tuple but its notation is different from the notation the authors use for vectors

\vskip+1ex
\noindent \dotfill
\vskip+1ex
%
\begin{quotation}
{\color{blue}
Thank you for the observation.  We worked hard to improve all the mathematical description of the method. Section FEATURE was modified as well as section PARAMETERS.  Section SIGNAL PLOT was revised and its mathematical terms were corrected.}
\end{quotation}
%
\vskip+1ex
\noindent \dotfill
\vskip+1ex

3.2) Moreover, there is something wrong in equation (4)

\vskip+1ex
\noindent \dotfill
\vskip+1ex
%
\begin{quotation}
{\color{blue}
We appreciate a lot your truthful comment.  Equation (4) is very important for us because it is the essence of the method.  %We based that equation on the description of how the histogram of oriented gradients is calculated published in Vedaldi and Szeliski, Computer Vision and Algorithms.  
We reviewed the equation, clarified their terms and we hope it is now more clear.
}
\end{quotation}
%
\vskip+1ex
\noindent \dotfill
\vskip+1ex

3.3) and something inconsistent between equations (7) and (8)

\vskip+1ex
\noindent \dotfill
\vskip+1ex
%
\begin{quotation}
{\color{blue}
We have revised the manuscript and corrected equations (7) and (8) which specify the ration between the signal properties and the SIFT's geometric parameters. 
We thank you for noticing this.
}
\end{quotation}
%
\vskip+1ex
\noindent \dotfill
\vskip+1ex

3.4) Also, the authors go back and forth the digital space and the continuous space and although there is an equivalence, the manuscript could benefit from a clearer explanation. 

\vskip+1ex
\noindent \dotfill
\vskip+1ex
%
\begin{quotation}
{\color{blue}
Thank you for the observation.  We modified all equations to emphasize the point that they are calculated in digital space. We also added that image gradients are also calculated using finite differences.
With this, we hope we have addressed your point.
 }
\end{quotation}
%
\vskip+1ex
\noindent \dotfill
\vskip+1ex

\subsection*{\ovalbox{Fourth Paragraph}}

4.1) Another issue that I find worrisome is the design of experiments.\\

\vskip+1ex
\noindent \dotfill
\vskip+1ex
%
\begin{quotation}
{\color{blue}
Thank you for the observation. 
We fully addressed your point.
 }
\end{quotation}
%
\vskip+1ex
\noindent \dotfill
\vskip+1ex

4.2) it seems that the number of subjects is rather low, it would be interesting to know why the authors decided to use such a subset.

\vskip+1ex
\noindent \dotfill
\vskip+1ex
%
\begin{quotation}
{\color{blue}
Thank you for your suggestion. The original dataset was published at the BNCI-Horizon website, which is very popular in BCI research.  Our approach emphasize the structure of the signal in terms of their morphological components in their visual description.% THIS IS TOUGH TO JUSTIFY 
In order to address this issue, we included a new dataset that we generated ourselves, on healthy subjects that we hope now improves the number of different subjects and that can help to validate the method.
}
\end{quotation}
%
\vskip+1ex
\noindent \dotfill
\vskip+1ex

4.3) Also, the authors speculate that their results are not that good because they did not take into account the latency and magnitude of the signals, there is previous evidence in the literature that the P300 suffers from these, and other, problems and, thus, one wonders why the authors did not anticipate such problems. 


\vskip+1ex
\noindent \dotfill
\vskip+1ex
%
\begin{quotation}
{\color{blue}
THIS IS EVEN TOUGHER.

ACA LE PUEDO PONER QUE HAY POCOS METODOS QUE PERMITAN ENGANCHAR BIEN QUE EL METODO


All these points were modified according to the suggestions.
We hope this is acceptable.
}
\end{quotation}
%
\vskip+1ex
\noindent \dotfill
\vskip+1ex


\subsection*{\ovalbox{Fifth Paragraph}}

5.1) it seems the authors assume that the readers have some previous deep knowledge of some topics such as that some channels should carry more or better information without a previous reference.


\vskip+1ex
\noindent \dotfill
\vskip+1ex
%
\begin{quotation}
{\color{blue}
The Results section was modified to clarify more precisely the expectation about the EEG channels where a visual P300 detection task would be obtained.  We added references and improved the section to clarify this point more thoughtfully.
We hope this is acceptable.
}
\end{quotation}
%
\vskip+1ex
\noindent \dotfill
\vskip+1ex

5.2)  issue has to do with the way the manuscript is written

\vskip+1ex
\noindent \dotfill
\vskip+1ex
%
\begin{quotation}
{\color{blue}
Thank you a lot for this important comment.  We re-read all the manuscript and asked to collegues to proofread the article.  They gave us interesting suggestions that helped us to improve different sections.  We would be very pleased if you find it now easier to read it.
}
\end{quotation}
%
\vskip+1ex
\noindent \dotfill
\vskip+1ex


5.3) sometimes is very difficult to follow the ideas the authors want to convey, an example can be found in the paragraph containing equation (5)


\vskip+1ex
\noindent \dotfill
\vskip+1ex
%
\begin{quotation}
{\color{blue}
Thank you very much for this information.  We reviewed completely that equation (it is now equation (6) and equation (7)) and we included a step-by-step procedure that describes the algorithm that we propose and we used.
}
\end{quotation}
%
\vskip+1ex
\noindent \dotfill
\vskip+1ex



\end{document}